\documentclass{beamer}


%%%%%%%%% PACKAGES %%%%%%%%%%%%%%%%%
\usepackage[utf8]{inputenc}
\usepackage[T1]{fontenc}
\usepackage{amsmath}
\usepackage{amssymb}
\usepackage{amsfonts}
\usepackage{alltt}
\renewcommand{\ttdefault}{txtt} % to resolve a problem with bold fonts in alltt

\usepackage{listings}
\lstset{language=haskell,basicstyle=\ttfamily}

\usepackage{proof}
\inferLineSkip=4pt  % increase spacing between lines; default is 2pt

\usepackage[table]{xcolor}
\usepackage[colorlinks,hyperindex,bookmarks,linkcolor=blue,citecolor=blue,urlcolor=blue]{hyperref}
\usepackage{todonotes}

\usepackage{fancyhdr}

%\usepackage{floatflt}
\usepackage{wrapfig}
\usepackage{caption}
\usepackage{subcaption}
\usepackage{framed}
\usepackage{multicol}

\makeatletter

\usepackage{babel}
\usepackage{graphicx}
\usepackage{theorem}

\usepackage{tikz}
\usetikzlibrary{trees}
\usetikzlibrary{positioning} 
\usepackage{tikzsymbols}


\makeatother

%%%%%%%%% GEOMETRY %%%%%%%%%%%%%%%%%
\addtolength{\topmargin}{-15mm}
\addtolength{\textheight}{25mm}
\addtolength{\oddsidemargin}{-20mm}
\setlength{\textwidth}{16cm}

%%%%%%%%% DECLS / DEFNS %%%%%%%%%%%%%%%%%

\usepackage{comment}
\specialcomment{solution}
{\todo[inline]{BEGIN SOLUTION}}
{\todo[inline]{END SOLUTION}}

{\theorembodyfont{\rmfamily} 
  \newtheorem{exo}{Exercise}
  \newtheorem{rem}{Remark}
}

% Macros for references
\newcommand{\polyref}[1]{polycopié, {#1}}

%%%%%%%%% END DECLS / DEFNS %%%%%%%%%%%%%%%%%

%%% Local Variables: 
%%% mode: latex
%%% coding: utf-8-unix
%%% End: 

% Theorems and definitions

% \newtheorem{definition}{Definition}
% \newtheorem{theorem}{Theorem}
% \newtheorem{lemma}{Lemma}
% \newtheorem{proposition}{Proposition}


% Definition of colors
\newcommand{\blue}[1]{{\color{blue}#1}}
\newcommand{\green}[1]{{\color{green}#1}}
\newcommand{\red}[1]{{\color{red}#1}}
\newcommand{\gray}[1]{{\color{gray}#1}}

% From MSCS file
\newcommand{\eg}{\textit{e.g.\ }}
\newcommand{\etal}{\textit{et al.\ }}
\newcommand{\etc}{\textit{etc}}
\newcommand{\ie}{\textit{i.e.\ }}
\newcommand{\viz}{\textit{viz.\ }}
\newcommand{\wrt}{\textit{w.r.t.\ }}
\newcommand{\lex}{\langle}
\newcommand{\rex}{\rangle}

% Own abbreviations
\newcommand{\secref}[1]{Section~\ref{#1}}
\newcommand{\secrefs}[1]{Sections~\ref{#1}}
\newcommand{\figref}[1]{Figure~\ref{#1}}
\newcommand{\figrefs}[1]{Figures~\ref{#1}}
\newcommand{\pgref}[1]{page~\pageref{#1}}
\newcommand{\theoremref}[1]{Theorem~\ref{#1}}
\newcommand{\theoremrefs}[1]{Theorems~\ref{#1}}
\newcommand{\lemmaref}[1]{Lemma~\ref{#1}}
\newcommand{\exampleref}[1]{Example~\ref{#1}}
\newcommand{\defref}[1]{Definition~\ref{#1}}

\newcommand{\figline}{\rule{\textwidth}{0.5pt}}


% Logique

\newcommand{\IMPL}[0]{\longrightarrow}
\newcommand{\IMPLL}[0]{\Longrightarrow} % another implication, to make
                                % a difference with reduction relations
\newcommand{\AND}[0]{\land}
\newcommand{\OR}[0]{\lor}
\newcommand{\NOT}[0]{\lnot}
\newcommand{\FALSE}[0]{\perp}
\newcommand{\TRUE}[0]{\top}
\newcommand{\IFF}[0]{\leftrightarrow}
\newcommand{\BIGAND}[1]{\bigwedge_{#1}}
\newcommand{\BIGOR}[1]{\bigvee_{#1}}
\newcommand{\BIGANDC}[2]{\bigwedge_{#1|#2}} % bigand with constraint
\newcommand{\BIGORC}[2]{\bigvee_{#1|#2}}    % bigor with constraint

\newcommand{\exgeq}[1]{\exists^{{\geq #1}}}
\newcommand{\exeq}[1]{\exists^{{= #1}}}
\newcommand{\exle}[1]{\exists^{{< #1}}}

% Remark macros for the authors

\newcommand{\remms}[2][]{\todo[color=green!40,#1]{MS: #2}}
\newcommand{\remre}[2][]{\todo[color=blue!40,#1]{RE: #2}}
\newcommand{\remjhb}[2][]{\todo[color=blue!20,#1]{JHB: #2}}


% Other

\newcommand{\smalltalcq}[0]{{\small small}-t{$\cal ALCQ$}}
\newcommand{\smalltalcqe}[0]{{\small small}-t{$\cal ALCQ$e}}
\newcommand{\trule}[0]{\xhookrightarrow}
\newcommand{\tableaurule}[1]{{\xhookrightarrow[]{#1}}}
\newcommand{\nodes}[1]{{\cal N}({#1})}
\newcommand{\trans}[1]{{\cal T}({#1})}
\newcommand{\transm}[1]{{\cal T'}({#1})}
\newcommand{\rconts}[1]{\llparenthesis #1 \rrparenthesis} %record contents
\newcommand{\rupd}[2]{{#1}\llparenthesis #2 \rrparenthesis} %record update

\newcommand{\eform}[0]{\mathit{eform}}
\newcommand{\form}[0]{\mathit{form}}
\newcommand{\free}[0]{\mathit{free}}
\newcommand{\exclprop}[0]{\stackrel{\times}{\longrightarrow}}

%%% Local Variables: 
%%% mode: latex
%%% TeX-master: "main"
%%% End: 


\title{L4 design}

\author{Martin Strecker}
\date{2021-03-02}


%======================================================================

\begin{document}


%======================================================================

\begin{frame}
  \titlepage
\end{frame}



%======================================================================
\section{Snapshot: Baby-L4 now}


%-------------------------------------------------------------
\begin{frame}[fragile]\frametitle{General structure of an L4 file}

  \begin{itemize}
  \item Lexicon
\begin{alltt}
\textbf{lexicon}
Business -> business_2 #from WordNet
Value -> value_1 
\end{alltt}
    
  \item List of class definitions (class $\approx$ type):
\begin{alltt}
\textbf{class} Business \{
      foo: Int
      bar: Bool -> (Int,Int)
\}

\textbf{class} LawRelatedService \textbf{extends} Business \{
\}
\end{alltt}

    
  \end{itemize}

\end{frame}


%-------------------------------------------------------------
\begin{frame}[fragile]\frametitle{General structure of an L4 file}

  \begin{itemize}
  \item Declarations:
\begin{alltt}
\textbf{decl} AssociatedWith: LegalPractitioner -> 
                      Appointment -> Bool
\textbf{decl} MayAcceptApp : LegalPractitioner -> 
                      Appointment -> Bool
\textbf{decl} ProhibitedBusiness : Business -> Bool
\end{alltt}
    
  \end{itemize}

\end{frame}

%-------------------------------------------------------------
\begin{frame}[fragile]\frametitle{General structure of an L4 file}


  \begin{itemize}
  \item Rules:
\begin{alltt}
\textbf{rule} <r1a>
\textbf{for} lpr: LegalPractitioner, app: Appointment
\textbf{if} (exists bsn : Business. 
         AssociatedWithAppB app bsn 
      \&\& IncompatibleDignity bsn)
\textbf{then} MustNotAcceptApp lpr app
\end{alltt}
  
\item Assertions / Goals to be proved:
\begin{alltt}
\textbf{assert} 
  exists lpr: LegalPractitioner. 
  exists app: Appointment. 
      MayAcceptApp lpr app
\end{alltt}
  \end{itemize}


\end{frame}

  
%======================================================================
\section{What to do with Baby-L4}

%-------------------------------------------------------------
\begin{frame}[fragile]\frametitle{Class definitions}

  \blue{Purpose:}
  \begin{itemize}
  \item Separation of data and ``rules''
  \item Used in type checking
  \end{itemize}
  
  \blue{What to do with it?}
  \begin{itemize}
  \item Parse data from data description languages:\\
    YAML, Jason, \dots\\
    \dots and check conformity with class defs
  \item Generate language stubs for OO languages
  \item Use in verification tools like Alloy
  \end{itemize}

\end{frame}

%-------------------------------------------------------------
\begin{frame}[fragile]\frametitle{Semantics of classes and fields}

  \blue{Mathematically:} Class = set of objects

  \blue{Pragmatically:}
  \begin{itemize}
  \item Fields corrsponding to relation declarations?
  \item Fields/ methods in the sense of OO languages?
  \end{itemize}
  Complementary, not incompatible views.

\end{frame}


%-------------------------------------------------------------
\begin{frame}[fragile]\frametitle{Semantics of classes and fields}

  \blue{Fields as function / relation declarations?}

\begin{alltt}
\textbf{class} LegalPractitioner \textbf{extends} Person \{
\}
\textbf{decl} AssociatedWith:
     LegalPractitioner -> Appointment -> Bool
\end{alltt}
  
the same as?:

\begin{alltt}
\textbf{class} LegalPractitioner \textbf{extends} Person \{
    AssociatedWith: Appointment -> Bool
\}
\end{alltt}

Functional / relational view: we write

\begin{alltt}
  forall lpr  : LegalPractitioner.
  exists app: Appointment.
     AssociatedWith lpr app
\end{alltt}

\end{frame}


%-------------------------------------------------------------
\begin{frame}[fragile]\frametitle{Semantics of classes and fields}

  \blue{Fields as components of a record}


  ``Relational'' view of components (as in Alloy)
\begin{alltt}
\textbf{class} LegalPractitioner \textbf{extends} Person \{
    salary: Int
\}
\end{alltt}

Understanding: \texttt{salary} is a relation\\
\texttt{LegalPractitioner} $\times$ \texttt{Int}

and \texttt{lpr.salary} is relation composition.

\end{frame}

%-------------------------------------------------------------
\begin{frame}[fragile]\frametitle{Semantics of classes and fields}

  \blue{Fields as components of a record}


  ``Functional'' view of components:
\begin{alltt}
\textbf{class} LegalPractitioner \textbf{extends} Person \{
      salary: Int
\}
\end{alltt}

For \texttt{lpr: LegalPractitioner}, one can write:\\
\texttt{lpr.salary}

which is syntactic sugar of \texttt{salary(lpr)}

\red{Downside:} under the relational interpretation, \texttt{lpr.salary} is
not uniquely determined $\leadsto$ cardinality annotations.

\end{frame}
  
%======================================================================
\section{Moving to L4}


%-------------------------------------------------------------
\begin{frame}[fragile]\frametitle{Rule modifiers}

  ... such as \texttt{subject to} clauses:

  \begin{itemize}
  \item annotations in rules 
  \item compilation of preconditions
  \end{itemize}


\end{frame}


%-------------------------------------------------------------
\begin{frame}[fragile]\frametitle{Deontic statements}

  Further investigations:

  \begin{itemize}
  \item Obligations as requirements on a module
  \item Permissions as requirements on its environment
  \item Composition of contracts
  \item Rely-Guarantee reaasoning
  \end{itemize}

\end{frame}



%-------------------------------------------------------------

\end{document}


%%% Local Variables: 
%%% mode: latex
%%% TeX-master: t
%%% coding: utf-8
%%% End: 
