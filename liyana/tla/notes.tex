% TeX ignores anything on a line after a %
% The next two lines define fonts for the title
\font\xmplbx = cmbx10 scaled \magstephalf
\font\xmplbxti = cmbxti10 scaled \magstephalf
% Now here's the title.
\leftline{\xmplbx Example 1:\quad\xmplbxti Entering simple text}
\vglue .5\baselineskip % skip an extra half line
It's easy to prepare ordinary text for \TeX\ since
\TeX\ usually doesn't care about how you break up your input into
lines. It treats the end of a line of text like a space.%
\footnote \dag{\TeX\ treats a tab like a space too, as we point
out in this {\it footnote}.} If you don't want a space there,
put a per%
cent sign (the comment character) at the end of the line.
\TeX\ ignores spaces at the start of a line, and treats more
than one space as equivalent to a single space,
even after a period. You indicate a new paragraph by
skipping a line (or more than one line).
When \TeX\ sees a period followed by a space (or the end of the
line, which is equivalent), it ordinarily assumes you've ended a
sentence and inserts a little extra space after the period. It
treats question marks and exclamation points the same way.
But \TeX's rules for handling periods sometimes need fine
tuning. \TeX\ assumes that a capital letter before a period
doesn't end the sentence, so you have to do something a little
different if, say, you're writing about DNA\null.
% The \null prevents TeX from perceiving the capital `A'
% as being next to the period.
It's a good idea to tie words together in references such as
``see Fig.~8'' and in names such as V.~I\null. Lenin and in
$\ldots$ so that \TeX\ will never split them in an awkward place
between two lines. (The three dots indicate an ellipsis.)
You should put quotations in pairs of left and right
single ``quotes'' so that you get the correct left and right
double quotation marks. ``For adjacent single and double
quotation marks, insert a `thinspace'\thinspace''. You can
get en-dashes--like so, and em-dashes---like so.
\bye % end the document